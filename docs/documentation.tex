\documentclass[12pt, a4paper]{article}

\usepackage[czech]{babel}
%\usepackage[IL2]{fontenc}
\usepackage[utf8]{inputenc}
\usepackage{enumitem}
\usepackage{parskip}
\usepackage{tocloft}
\usepackage{multicol}
\usepackage[hidelinks]{hyperref}
\usepackage{graphicx}
\usepackage{float}
\usepackage{listings}
\usepackage{xcolor}

% Define code style for C
\lstset{
    language=C,
    basicstyle=\ttfamily,
    keywordstyle=\color{blue},
    stringstyle=\color{red},
    commentstyle=\color{green},
    breaklines=true
    frame=shadowbox,
    numbers=left,
    numberstyle=\small,
    stepnumber=1,
    numbersep=5pt,
    showstringspaces=false,
    tabsize=4,
    captionpos=b,
}


\begin{document}

%Uvodni strana
\begin{titlepage}
    \includegraphics[width=0.75\textwidth]{img/fav.png}
    \begin{center}
        
        
        \vspace{2cm}
        
        \Huge
        Semestrální práce KIV/ZOS
        
        \vspace{1cm}
        
        \LARGE
        Souborový systém založený na pseudoFAT       
        \vfill
        
        \vspace{0.5cm}
        
        \normalsize
        \raggedright
        Student:        Adam Míka \\
        Osobní číslo:   A22B0319P \\
        Email:          mikaa@students.zcu.cz \\
        Datum:          5. prosince 2024
        \vspace{0.2cm}
        
    \end{center}
\end{titlepage}

%tečky v obsahu
\renewcommand{\cftsecleader}{\cftdotfill{\cftdotsep}}
\renewcommand{\cftsubsecleader}{\cftdotfill{\cftdotsep}}
\renewcommand{\cftsubsubsecleader}{\cftdotfill{\cftdotsep}}

%obsah
\setcounter{page}{2}
\tableofcontents
\listoffigures
\lstlistoflistings
\pagebreak


\section{Zadání}
\large
\textbf{Hlavní cíle:}
\normalsize
Zbytek zadání zde: \href{https://portal.zcu.cz/CoursewarePortlets2/DownloadDokumentu?id=238432}{\textcolor[RGB]{20,20,200}{\underline{\textit{PDF}}}}


\section{Analýza úlohy}

Cílem semestrální práce je implementace zjednodušeného souborového systému na bázi pseudoFAT, který umožní uživatelům provádět základní operace se soubory a adresáři. Zadané úlohy lze rozdělit do několika hlavních kategorií podle funkcionalit, které systém poskytuje. V této sekci analyzujeme požadavky na jednotlivé příkazy a jejich význam pro souborový systém.

\subsection{Základní operace}

\begin{itemize}
    \item \textbf{Kopírování souboru (cp)}: Tento příkaz umožňuje kopírovat soubor z jednoho umístění (zdroj) do druhého (cíl). Součástí implementace je nutnost validace cesty k cíli i zdroji. Příkaz musí detekovat chyby, jako například neexistující zdroj nebo cíl.
    
    \item \textbf{Přesunutí nebo přejmenování souboru (mv)}: Tento příkaz umožňuje přesunout soubor z jedné cesty do jiné, nebo přejmenovat soubor. Klíčovou částí je validace, zda cílová cesta existuje a zda nedochází ke konfliktu názvů.
    
    \item \textbf{Odstranění souboru (rm)}: Umožňuje mazat konkrétní soubory. Důležité je validovat, zda je mazán skutečně soubor (ne adresář), a ověřit existenci souboru.
    
    \item \textbf{Vytvoření adresáře (mkdir)}: Slouží k vytvoření nového adresáře. Příkaz musí ověřit, zda již neexistuje adresář nebo soubor se stejným názvem v cílové cestě.
    
    \item \textbf{Odstranění prázdného adresáře (rmdir)}: Tento příkaz odstraňuje pouze prázdné adresáře. Je nutné kontrolovat, zda je adresář skutečně prázdný, a odlišit situaci, kdy je na vstupu zadán soubor místo adresáře.
\end{itemize}

\subsection{Rozšířené operace}

\begin{itemize}
    \item \textbf{Načítání příkazů ze souboru (load)}: Tento příkaz umožňuje provést příkazy uložené v textovém souboru. Formát vstupního souboru musí být validní a každý příkaz by měl být zpracován sekvenčně.
    
    \item \textbf{Formátování souborového systému (format)}: Příkaz vytvoří nový souborový systém na zadané velikosti. Při implementaci je třeba zohlednit inicializaci tabulek FAT a alokaci prostoru pro datové bloky.
    
    \item \textbf{Kontrola integrity souborového systému (check)}: Zajišťuje ověření konzistence struktury souborového systému, jako jsou chybné nebo ztracené bloky, nekonzistentní tabulky FAT a správnost datových struktur.
    
    \item \textbf{Poškození souborového systému (bug)}: Simuluje poškození systému, aby bylo možné testovat příkaz \textit{check}.
\end{itemize}

\subsection{Hodnocení a očekávané výsledky}

Při implementaci a testování systému je nutné dodržet následující požadavky:
\begin{itemize}
    \item Funkční a korektní zpracování všech příkazů.
    \item Validace vstupů a robustní detekce chybových stavů.
    \item Testování chování systému při chybách (např. použití \textit{bug} a následné ověření příkazem \textit{check}).
    \item Dokumentace výsledků a výpisů příkazů v předepsaném formátu.
\end{itemize}

\subsection{Funkce FAT tabulky}

FAT (File Allocation Table) je jádrem správy souborového systému a hraje zásadní roli při sledování alokace dat na disku. Princip FAT tabulky spočívá v udržování seznamu propojených bloků dat (clusterů), které tvoří jednotlivé soubory či složky. Každý cluster má odpovídající záznam v tabulce, kde je uvedeno, zda:

\begin{itemize}
    \item je cluster součástí souboru/složky a na který následující cluster odkazuje,
    \item je cluster označen jako konec souboru (\texttt{FAT\_FILE\_END}),
    \item je cluster volný pro použití (\texttt{FAT\_UNUSED}),
    \item je cluster poškozený (\texttt{FAT\_BAD\_CLUSTER}).
\end{itemize}

Tento systém umožňuje efektivní navigaci mezi jednotlivými částmi dat souboru a optimalizuje správu volného prostoru na disku.

\subsection{Vlastnosti FAT tabulky v systému}

\begin{itemize}
    \item \textbf{Jednoduchost:} FAT tabulka je implementována jako jednorozměrné pole čísel, kde index pole odpovídá číslu clusteru. Hodnota uložená na daném indexu určuje další cluster v řetězci nebo speciální stav (např. \texttt{FAT\_FILE\_END}).
    \item \textbf{Flexibilita:} Soubory mohou být uložené v nepropojených clusterech. Pokud je soubor fragmentovaný, tabulka umožňuje jeho rekonstrukci díky propojení mezi clustery.
    \item \textbf{Omezení:} Fragmentace může vést ke sníženému výkonu při práci s velkými soubory, protože systém musí neustále přeskakovat mezi clustery.
\end{itemize}

\subsection{Správa poškození a kontrola integrity}

Pro kontrolu integrity souborového systému je využíván příkaz \texttt{check}, který identifikuje poškozené clustery (\texttt{FAT\_BAD\_CLUSTER}) a zjišťuje, zda nejsou součástí žádného řetězce. Příkaz \texttt{bug} umožňuje simulaci chyb tím, že označí cluster jako poškozený. Takto je možné testovat robustnost systému a funkčnost kontrolních mechanismů.

\subsection{Ilustrace funkčnosti}

Například soubor rozdělený do dvou clusterů:

\begin{itemize}
    \item \texttt{FAT[0] = 1} (první cluster ukazuje na druhý),
    \item \texttt{FAT[1] = FAT\_FILE\_END} (konec souboru).
\end{itemize}

Tento přístup zajišťuje jednoduchou správu dat i v případě, že disk obsahuje fragmentované soubory. Pokud by se cluster označený jako \texttt{FAT\_BAD\_CLUSTER} stal součástí řetězce, příkaz \texttt{check} tuto chybu identifikuje a upozorní na ni.

\newpage
\section{Implementace systému PseudoFAT}
Tato sekce stručně popisuje klíčové metody implementace systému PseudoFAT. Každá metoda je uvedena s popisem svého účelu, hlavní funkcionality a výstupu.

\subsection{Inicializace systému}
\textbf{Metoda: Konstruktor \texttt{PseudoFAT::PseudoFAT(const std::string \&file)}} \\
\textbf{Účel:} Zajišťuje načtení existujícího souborového systému nebo vytvoření nového při absenci souboru. \\
\textbf{Funkce:}
\begin{itemize}
    \item Zkontroluje, zda zadaný soubor existuje.
    \item Pokud neexistuje, umožní uživateli provést formátování pomocí příkazu \texttt{format}.
    \item Pokud existuje, načte struktury systému a aktualizuje ukazatel na další volné ID.
\end{itemize}
\textbf{Výstup:} Úspěšné načtení nebo inicializace systému.

\subsection{Formátování disku}
\textbf{Metoda: \texttt{PseudoFAT::formatDisk(const std::string \&sizeStr)}} \\
\textbf{Účel:} Inicializuje nový souborový systém s definovanou velikostí. \\
\textbf{Funkce:}
\begin{itemize}
    \item Vypočítá a inicializuje FAT tabulky.
    \item Vytvoří datový soubor s nulovými hodnotami o zadané velikosti.
    \item Přidá kořenový adresář do struktury systému.
    \item Uloží stav systému do souboru.
\end{itemize}
\textbf{Výstup:} OK nebo chybová zpráva při selhání vytvoření souboru.

\subsection{Práce s adresáři}
\textbf{Metoda: \texttt{PseudoFAT::createDirectory(const std::string \&path)}} \\
\textbf{Účel:} Vytvoří nový adresář na zadané cestě. \\
\textbf{Funkce:}
\begin{itemize}
    \item Ověří, zda cesta obsahuje pouze jedno jméno adresáře.
    \item Zkontroluje, zda v cílovém adresáři již neexistuje soubor nebo adresář se stejným názvem.
    \item Přidá nový adresář do cílového adresáře a aktualizuje strukturu systému.
\end{itemize}
\textbf{Výstup:} OK nebo chybová zpráva při neplatné cestě nebo jménu.

\subsection{Kontrola integrity systému}
\textbf{Metoda: \texttt{PseudoFAT::check()}} \\
\textbf{Účel:} Zajišťuje, že FAT tabulky a datové struktury systému nejsou poškozené. \\
\textbf{Funkce:}
\begin{itemize}
    \item Prochází FAT tabulku a kontroluje neplatné clustery (např. \texttt{FAT\_BAD\_CLUSTER}).
    \item Hledá sirotčí clustery, které nejsou propojené s žádným souborem.
\end{itemize}
\textbf{Výstup:} Zpráva o stavu systému – buď „No corruption detected“, nebo podrobnosti o nalezených chybách.

\subsection{Mazání adresářů}
\textbf{Metoda: \texttt{PseudoFAT::rmdir(const std::string \&path)}} \\
\textbf{Účel:} Odstraní prázdný adresář na zadané cestě. \\
\textbf{Funkce:}
\begin{itemize}
    \item Ověří, zda zadaná cesta odkazuje na adresář.
    \item Zkontroluje, zda je adresář prázdný.
    \item Odstraní adresář z rodičovské struktury a uloží změny do souboru.
\end{itemize}
\textbf{Výstup:} OK nebo chybová zpráva při neplatné cestě nebo neúspěšném mazání.

\subsection{Přesun a kopírování souborů}
\textbf{Metoda: \texttt{PseudoFAT::mv(const std::string \&srcPath, const std::string \&destPath)}} \\
\textbf{Účel:} Přesune nebo přejmenuje soubor. \\
\textbf{Funkce:}
\begin{itemize}
    \item Ověří existenci zdrojového souboru a cílového adresáře.
    \item Zkontroluje konflikty názvů v cílovém adresáři.
    \item Aktualizuje informace o souboru a uloží změny.
\end{itemize}
\textbf{Výstup:} OK nebo chybová zpráva.

\textbf{Metoda: \texttt{PseudoFAT::cp(const std::string \&srcPath, const std::string \&destPath)}} \\
\textbf{Účel:} Kopíruje soubor do nového umístění. \\
\textbf{Funkce:}
\begin{itemize}
    \item Alokuje nové clustery pro kopii souboru.
    \item Zkopíruje data ze zdrojových clusterů do nových.
    \item Přidá kopii souboru do cílového adresáře.
\end{itemize}
\textbf{Výstup:} OK nebo chybová zpráva.


\section{Uživatelská příručka}

\subsection{Překlad programu}
Pro překlad programu použijte následující příkazy:
\begin{verbatim}
make clean
make
\end{verbatim}

\subsection{Spuštění programu}
Program se spouští pomocí:
\begin{verbatim}
./main <název_souboru>
\end{verbatim}
Například:
\begin{verbatim}
./main fileSystem.dat
\end{verbatim}

\subsubsection{Nový souborový systém}
Pokud soubor \texttt{fileSystem.dat} neexistuje, je nutné jej nejprve naformátovat. Použijte příkaz:
\begin{verbatim}
format 600MB
\end{verbatim}

\subsubsection{Otevření existujícího souborového systému}
Pokud soubor již existuje a byl dříve použit, program jej načte a umožní okamžitou práci s příkazy.

\subsubsection{Chyba souborového systému}
Pokud soubor obsahuje chybu, například \texttt{File bad cluster}, mohou být použity pouze následující příkazy:
\begin{itemize}
    \item \texttt{format} – Opětovné naformátování souborového systému.
    \item \texttt{check} – Kontrola integrity souborového systému.
    \item \texttt{exit} – Ukončení programu.
\end{itemize}

\subsection{Dostupné příkazy}
Níže je uveden seznam příkazů a jejich popis.

\subsubsection{Základní příkazy}
\begin{itemize}
    \item \textbf{\texttt{format <velikost>}} – Formátuje souborový systém na danou velikost.
    \item \textbf{\texttt{exit}} – Ukončí program.
\end{itemize}

\subsubsection{Správa adresářů}
\begin{itemize}
    \item \textbf{\texttt{mkdir <cesta>}} – Vytvoří nový adresář.
    \item \textbf{\texttt{rmdir <cesta>}} – Smaže prázdný adresář.
    \item \textbf{\texttt{cd <cesta>}} – Změní aktuální pracovní adresář.
    \item \textbf{\texttt{ls <cesta>}} – Vypíše obsah adresáře.
\end{itemize}

\subsubsection{Práce se soubory}
\begin{itemize}
    \item \textbf{\texttt{incp <zdroj> <cíl>}} – Importuje soubor z pevného disku do systému.
    \item \textbf{\texttt{outcp <zdroj> <cíl>}} – Exportuje soubor ze systému na pevný disk.
    \item \textbf{\texttt{rm <cesta>}} – Smaže soubor.
    \item \textbf{\texttt{mv <zdroj> <cíl>}} – Přesune nebo přejmenuje soubor/adresář.
    \item \textbf{\texttt{cp <zdroj> <cíl>}} – Zkopíruje soubor/adresář.
\end{itemize}

\subsubsection{Další příkazy}
\begin{itemize}
    \item \textbf{\texttt{cat <cesta>}} – Vypíše obsah souboru.
    \item \textbf{\texttt{info <cesta>}} – Zobrazí informace o souboru nebo adresáři.
    \item \textbf{\texttt{load <soubor>}} – Načte příkazy ze souboru a provede je.
    \item \textbf{\texttt{check}} – Zkontroluje integritu souborového systému.
    \item \textbf{\texttt{bug <cesta>}} – Poškodí souborový systém pro testování.
\end{itemize}

\subsection{Chybová hlášení}
\begin{itemize}
    \item \texttt{INVALID PATH} – Neplatná cesta.
    \item \texttt{PATH NOT FOUND} – Cesta nebyla nalezena.
    \item \texttt{FILE NOT FOUND} – Soubor nebyl nalezen.
    \item \texttt{SAME NAME} – Název již existuje.
    \item \texttt{NOT ENOUGH SPACE} – Nedostatek místa pro operaci.
\end{itemize}


\section{Závěr}
Popisované metody demonstrují klíčové části implementace systému PseudoFAT. Každá metoda je navržena tak, aby splňovala požadavky na správu souborového systému a zároveň zajišťovala robustní detekci a opravu chyb.

\end{document}
